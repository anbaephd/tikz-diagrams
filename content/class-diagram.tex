\section{Class Diagram}



\begin{figure}[th!]
\centering
\begin{tikzpicture}
\begin{umlpackage}{p}
\begin{umlpackage}{sp1}
\umlclass[template=T]{A}{
  n : uint \\ t : float
}{}
\umlclass[y=-3]{B}{
  d : double
}{
  \umlvirt{setB(b : B) : void} \\ getB() : B}
\end{umlpackage}
\begin{umlpackage}[x=10,y=-6]{sp2}
\umlinterface{C}{
  n : uint \\ s : string
}{}
\end{umlpackage}
\umlclass[x=2,y=-10]{D}{
  n : uint
  }{}
\end{umlpackage}

\umlassoc[geometry=-|-, arg1=tata, mult1=*, pos1=0.3, arg2=toto, mult2=1, pos2=2.9, align2=left]{C}{B}
\umlunicompo[geometry=-|, arg=titi, mult=*, pos=1.7, stereo=vector]{D}{C}
\umlimport[geometry=|-, anchors=90 and 50, name=import]{sp2}{sp1}
\umlaggreg[arg=tutu, mult=1, pos=0.8, angle1=30, angle2=60, loopsize=2cm]{D}{D}
\umlinherit[geometry=-|]{D}{B}
\umlnote[x=2.5,y=-6, width=3cm]{B}{Je suis une note qui concerne la classe B}
\umlnote[x=7.5,y=-2]{import-2}{Je suis une note qui concerne la relation d'import}
\end{tikzpicture}
\caption{bla bla bla}
\end{figure}
